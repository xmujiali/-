\chapter{on \LaTeX}

\section{\LaTeX{}基本命令}

\subsection{使用\LaTeX{}抄录代码}
抄录代码可以使用lstlisting环境,抄录C代码的办法如下:
\begin{verbatim}
\begin{lstlisting}[language=C]
int main(int argc, char ** argv)
{
printf("Hello world!\n");
return 0;
}
\end{lstlisting}
\end{verbatim}

\noindent{}其显示效果如下:
\begin{lstlisting}[language=C]
int main(int argc, char ** argv)
{
printf("Hello world!\n");
return 0;
}
\end{lstlisting}


\noindent{}抄录Python如下:
\begin{verbatim}
\begin{lstlisting}[language=Python]
from pyx import *
g = graph.graphxy(width=8)
g.plot(graph.data.function("y(x)=sin(x)/x", min=-15, max=15))
g.writePDFfile("function")
print r'\includegraphics{function}'
\end{lstlisting}
\end{verbatim}

\noindent{}其显示效果如下:
\begin{lstlisting}[language=Python]
from pyx import *
g = graph.graphxy(width=8)
g.plot(graph.data.function("y(x)=sin(x)/x", min=-15, max=15))
g.writePDFfile("function")
print r'\includegraphics{function}'
\end{lstlisting}


\section{字体}


\section{版面设计}


\section{标题}


\section{表格}


\section{列表}


\section{数学式}


\section{插图}


\section{正文工具}


\section{浮动体}


\section{幻灯片:beamer}

\subsection{tableofcontents}
%1、列目录时,隐藏所有的小节
%\begin{lstlisting}[language=Latex]
%\tableofcontents[hideallsubsections]
%\end{lstlisting}

%2、自动压缩,以显示全部内容
%\begin{lstlisting}[language=LaTeX]
%\begin{frame}[shrink]
%\end{lstlisting}
%
%3、在每一节(或小节)前增加目录
%命令 \AtBeginSection[]{} 和 \AtBeginSubsection[]{} 。比如下面两个命令可以实现在每一节前显示文档目录,隐藏所有小节标题,并高亮当前节标题,而在每一小节前显示的目录中,只会显示本节的小节标题,并高亮当前小节标题,其他节的小节标题不显示,只显示节标题。
%
%
%\begin{lstlisting}[language=LaTeX]
%\AtBeginSection[]
%{
%	
%	\begin{frame}
%		\tableofcontents[currentsection,hideallsubsections]
%	\end{frame}
%	
%	1
%	2
%	3
%	
%}
%
%\AtBeginSubsection[]
%{
%	
%	\begin{frame}[shrink]
%		\tableofcontents[sectionstyle=show/shaded,subsectionstyle=show/shaded/hide]
%	\end{frame}
%	
%	1
%	2
%	3
%	
%}
%\end{lstlisting}
%
%4、frametitle的两种写法
%\begin{lstlisting}[language=LaTeX]
%\begin{frame}
%	
%	\frametitle{标题}
%	
%	1
%	
%\end{frame}
%
%\begin{frame}{标题}
%	
%\end{frame}
%\end{lstlisting}
%btw: 其实这些技巧在手册中都有,只是手册太长,一直没有完整的读过。结果每次都有新发现。喜欢Beamer,不单因为Beamer确实比较方便、漂亮,更重要的是Beamer的作者对于幻灯片的制作和使用有自己独到的见解。内容的组织,颜色的搭配,时间的控制,都有所涉及。让人感觉非常专业,自然产生一种信任。

