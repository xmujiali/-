\chapter{Linux配置}
\section{修改启动级别}
该方案是过时的!!!

安装Linux时如果选择安装了图形界面,系统会启动后会默认启动Xwindows,如不希望如此可做如下操作:

以root用户修改/etc/inittab,找到id:x:initdefault:一行,x=3为文本方式 x=5为Xwindow方式,重启机器即可生效

\section{添加用户和用户组}


\begin{itemize}
\item{建用户}\\
adduser phpq    //新建phpq用户\\
passwd phpq     //给phpq用户设置密码


\item{建工作组}\\
groupadd test				//新建test工作组


\item{新建用户同时增加工作组}\\
useradd -g test phpq		//新建phpq用户并增加到test工作组\\
参数说明:-g 所属组 -d 家目录 -s 所用的SHELL类型


\item{给已有的用户增加工作组}\\
usermod -G groupname username\\
或\\
gpasswd -a user groupadd


\item{临时关闭}\\
在/etc/shadow文件中属于该用户的行的第二个字段(密码)前面加上*就可以了。想恢复该用户,去掉*即可。
或者使用如下命令关闭用户账号:\\
passwd peter –l\\
重新释放:\\
passwd peter –u


\item{永久性删除用户账号}\\
userdel peter\\
groupdel peter\\
usermod –G peter peter   // 强制删除该用户的主目录和主目录下的所有文件和子目录


\item{从组中删除用户}\\
编辑/etc/group 找到GROUP1那一行,删除 A。或者用命令:\\
gpasswd -d A GROUP


\item{显示用户信息}\\
id user\\
cat /etc/passwd
\end{itemize}









\chapter{Linux 网络配置}
\section{SSH自动掉线的解决办法}
SSH连接总是隔一段时间没有输入时就断开,解决办法如下:
\begin{itemize}
\item 服务端配置
\begin{lstlisting}[language=bash]
sudo vi /etc/ssh/sshd_config
ClientAliveInterval 60  # 服务端主动向客户端请求响应的间隔
ClientAliveCountMax 10  # 服务器发出请求后客户端没有响应的次数达到一定值就自动断开
sudo restart ssh
\end{lstlisting}

\item 客户端配置
\begin{lstlisting}[language=bash]
sudo vi /etc/ssh/ssh_config  # 或 vi ~/.ssh/config
TCPKeepAlive=yes
ServerAliveInterval 60   # 客户端主动向服务端请求响应的间隔
\end{lstlisting}

\item ssh时添加选项:
\begin{lstlisting}[language=bash]
ssh -i <key-file> -o StrictHostKeyChecking=no -o TCPKeepAlive=yes -o ServerAliveInterval=30 ubuntu@<ip>
\end{lstlisting}
\end{itemize}
上面方式任选一种,推荐客户端配置方式。


\section{通过sshfs将远程服务器的文件挂载到本地}
假定用户jli可以通过SSH连接到远程服务器。
为避免反复使用scp交换文件,可以直接把远程服务器上的目录挂载到本地。具体方案如下:
\begin{itemize}
\item 安装sshfs
\begin{lstlisting}[language=bash]
sudo apt install sshfs
# 要检查fuse组是否存在
cat /etc/group | grep 'fuse'
# 如果该组不存在,须创建fuse组
sudo groupadd fuse
# 如果用户jli不在fuse组中,将其加入
sudo usermod -a -G fuse jli
# 搞完后最好重新登录下
\end{lstlisting}

\item 安装sshfs
\begin{lstlisting}[language=bash]
# sshfs语法如下:
sshfs [user@]host:[directory] mountpoint [options]
# 我要把服务器chem174上的/home/jli目录
# 挂载到本地的/home/jli/remote/174下:
sshfs jli@chem174:/home/ /home/jli/remote/174
\end{lstlisting}
搞定!就像操作本地文件操作远程文件吧!
\end{itemize}

\chapter{Linux环境下编译、开发环境配置}


\chapter{CentOS相关}
\section{CentOS上网}
在RHEL或者CentOS等Redhat系的Linux系统里,跟网络有关的主要设置文件如下:
\begin{verbatim}
/etc/host.conf              # 配置域名服务客户端的控制文件
/etc/hosts                  # 完成主机名映射为IP地址的功能
/etc/resolv.conf            # 域名服务客户端的配置文件,用于指定域名服务器的位置
/etc/sysconfig/network      # 包含了主机最基本的网络信息,用于系统启动
/etc/sysconfig/network-script/      #系统启动时初始化网络的一些信息
/etc/xinetd.conf            # 定义了由超级进程xinetd启动的网络服务
/etc/networks               # 完成域名与网络地址的映射
/etc/protocols              # 设定了主机使用的协议以及各个协议的协议号
/etc/services               # 设定主机的不同端口的网络服务
\end{verbatim}



开启,关闭eth0网卡,注意网卡的名字可能不一样\\
ifup    eth0\\
ifdown  eth0


CentOS 查询IP地址,输入下面的命令:\\
ip -4 add


CentOS 打开/关闭网络连接,输入下面的命令:\\
ifup ens192 ifdw ens192


CentOS 重启网络服务,输入下面的命令:\\
systemctl retart network.service\\
或\\
systemctl restart network


CentOS  启动网络服务,输入下面的命令:\\
systemctl start network.service\\
或\\
systemctl start network


CentOS  停止网络服务,输入命令:\\
systemctl stop network.service\\
或\\
systemctl stop network

下面逐个文件说明配置方法:
\begin{itemize}
\item{/etc/host.conf}\\
文件的默认信息如下:\\ 
\begin{verbatim}
multi on           # 允许主机拥有多个IP地址
order hosts,bind   # 主机名解析顺序,即本地解析,DNS域名解析的顺序
\end{verbatim}
这个文件一般不需要我们修改,默认的解析顺序是本地解析,DNS服务器解析,也就是说在本系统里对于一个主机名首先进行本地解析,如果本地解析没有,然后进行DNS服务器解析。

\item{/etc/hosts}\\
文件默认的内容大概如下:\\
127.0.0.1   butbueatiful  localhost.localdomain  localhost\\
::1         localhost6.localdomain6 localhost6\\
可见,默认的情况是本机ip和本机一些主机名的对应关系,第一行是ipv4信息,第二行是ipv6信息,如果用不上ipv6本机解析,一般把该行注释掉。
第一行的解析效果是,butbueatiful localhost.localdomain localhost都会被解析成127.0.0.1,我们可以用ping试试。
\begin{verbatim}
[root@butbueatiful ~]# ping -c 3 butbueatiful
PING butbueatiful (127.0.0.1) 56(84) bytes of data.
64 bytes from butbueatiful (127.0.0.1): icmp_seq=1 ttl=64 time=0.061 ms
64 bytes from butbueatiful (127.0.0.1): icmp_seq=2 ttl=64 time=0.052 ms
64 bytes from butbueatiful (127.0.0.1): icmp_seq=3 ttl=64 time=0.051 ms

--- butbueatiful ping statistics ---
3 packets transmitted, 3 received, 0% packet loss, time 1999ms
rtt min/avg/max/mdev = 0.051/0.054/0.061/0.009 ms

[root@butbueatiful ~]# ping -c 3 localhost.localdomain
PING butbueatiful (127.0.0.1) 56(84) bytes of data.
64 bytes from butbueatiful (127.0.0.1): icmp_seq=1 ttl=64 time=0.055 ms
64 bytes from butbueatiful (127.0.0.1): icmp_seq=2 ttl=64 time=0.035 ms
64 bytes from butbueatiful (127.0.0.1): icmp_seq=3 ttl=64 time=0.050 ms

--- butbueatiful ping statistics ---
3 packets transmitted, 3 received, 0% packet loss, time 1999ms
rtt min/avg/max/mdev = 0.035/0.046/0.055/0.011 ms
\end{verbatim}
看到上面的结果,你可能会问为什么ping localhost.localdomain的时候,下面显示的是却是butbueatiful,这是因为第一个主机名butbueatiful后面的那些主机名其实都是butbueatiful的主机别名。


如果我们要追加新的本地解析,比如我们希望在我们的机器里把yyyy.com和www.yyyy.com都解析成192.168.0.100,那么就追加如下一句即可:\\
192.168.0.100 yyyy.com www.yyyy.com\\
同样,在这里,www.yyyy.com是yyyy.com的主机别名。\\
如果你仔细一想,会发现,其实这个文件是很危险的,如果有人恶意修改了你这个文件,比如把淘宝的网站域名解析到了他的钓鱼网站,那你就要中招了。


\item{/etc/resolv.conf}\\
指定域名解析的DNS服务器IP等信息, 配置参数一般接触到的有4个:\\
\begin{verbatim}
nameserver    # 指定DNS服务器的IP地址
domain        # 定义本地域名信息
search        # 定义域名的搜索列表
sortlist      # 对gethostbyname返回的地址进行排序
\end{verbatim}
但是最常用的配置参数是nameserver,其他的可以不设置,这个参数指定了DNS服务器的IP地址,如果设置不正确,就无法进行正常的域名解析。\\
一般来说,推荐设置2个DNS服务器,比如我们用google的免费DNS服务器,那么该文件的设置内容如下:\\
nameserver 8.8.8.8\\
nameserver 8.8.4.4\\
同样,这个文件也是危险的,如果被人恶意改成了他自己的DNS服务器,他就可以为所欲为的控制你通过域名访问的每个目的地了,这就是常说的DNS劫持。


\item{/etc/sysconfig/network}\\
典型的配置如下:
\begin{verbatim}
NETWORKING=yes
NETWORKING_IPV6=no
HOSTNAME=butbueatiful
GATEWAY=192.168.0.1

参数简要解释:
NETWORK             # 设置网络是否有效,yes有效,no无效
NETWORKING_IPV6     # 设置ipv6网络是否有效,yes有效,no无效
HOSTNAME            # 设置服务器的主机名,最好和/etc/hosts里设置一样,否则在使用一些程序的时候会有问题。
GATEWAY             # 指定默认网关IP
\end{verbatim}

\item{ifcfg-ethX}\\
设置对应网口的IP等信息, 比如第一个网口, 那么就是/etc/sysconfig/network-scripts/ifcfg-eth0,配置例子:\\
\begin{verbatim}
参数简要解释:\\
TYPE=Ethernet           #类型=以太网络\\
PROXY_METHOD=none       #代理模式\\
BROWSER_ONLY=no\\
BOOTPROTO=none          #开机协议\\
DEFROUTE=yes\\
IPV4_FAILURE_FATAL=no\\
IPV6INIT=yes\\
IPV6_AUTOCONF=yes\\
IPV6_DEFROUTE=yes\\
IPV6_FAILURE_FATAL=no\\
IPV6_ADDR_GEN_MODE=stable-privacy\\
NAME=ens32\\
UUID=0e6d72a7-8a6c-43ac-aef2-25d165562fd0\\
DEVICE=ens32            #设备\\
ONBOOT=yes              #启动或者重启网络时,是否启动该设备,yes是启动,no是不启动\\
IPADDR=192.168.1.15     #IP地址\\
PREFIX=24               #子网掩码\\
GATEWAY=192.168.1.1     #网关\\
DNS1=192.168.1.2        #DNS服务器地址\\
IPV6_PRIVACY=no         #IPV6协议\\

说明如下:\\
DEVICE        设备名,不要自己乱改,和文件ifcfg-ethX里的ethX要一致\\
BROADCAST     广播地址\\
HWADDR        物理地址,这个你不要乱改\\
IPADDR        IP地址\\
NETMASK       子网掩码\\
ONBOOT        启动或者重启网络时,是否启动该设备,yes是启动,no是不启动\\
BOOTPROTO     开机协议,最常见的三个参数如下:\\
static(静态IP)\\
none(不指定,设置固定ip的情况,这个也行,但是如果要设定多网口绑定bond的时候,必须设成none)\\
dhcp(动态获得IP相关信息)
\end{verbatim}

\item{route-ethX}\\
比如第一个网口eth0的路由信息,那么就是/etc/sysconfig/network-scripts/route-eth0:\\
比如我们现在有这样一个需求,通过eth0去网络172.17.27.0/24不走默认路由,需要走192.168.0.254,那么我们第一反应,肯定是用route命令追加路由信息:\\
\begin{verbatim}
[root@butbueatiful ~]# route add -net 172.17.27.0 netmask 255.255.255.0 gw 192.168.0.254 dev eth0
可是,你没意识到的是,这样只是动态追加的而已,重启网络后,路由信息就消失了,所以需要设置静态路由,这时候就要设置/etc/sysconfig/network-scripts/route-eth0文件了,如果没有该文件,你就新建一个:
[root@butbueatiful ~]# vi /etc/sysconfig/network-scripts/route-eth0
#追加
172.17.27.0/24via 192.168.0.254
这下即使重启网络,重启系统,该路由也会自动加载,当然了,如果你没有这样的需要,那么这个文件就没必要创建和配置了。
\end{verbatim}

\end{itemize}


% 二、常用的网络配置
 
% 伴随着时间的推移Red Hat公司推出了RHEL6.2,随后CentOS也紧随其后退出了CentOS6.2。新的系统中厂商加入了大量虚拟化及云计算的元素,同时对于细节的改变也不少,这里我们仅对新系统中的网络参数做以详尽说明。
 
% Linux中网络参数大致包含以下内容:
 
% IP地址
% 子网掩码
% 网关
% DNS服务器
% 主机名
 
% 历来Linux系统中修改这些参数的方式通常有:命令、文件两种。其中通过命令设置可以立即生效但重启后将失效,通过文件修改实现永久生效,但不会立即生效。
 
% 首先我们来看看命令的方式:
 
% ifconfig:查看与设置IP地址、子网掩码
% hostname:查看与设置主机名
% route:     查看与设置路由信息(默认网关等)
 
% 通过文件的方式修改:
 
% /etc/sysconfig/network-scripts/ifcfg-设备名(通常为ifcfg-eth0)
% /etc/sysconfig/network
% /etc/resolv.conf文件:设置DNS服务器
 
% 以上种种这些方式可以同时在5.0与6.0系统中实现,但6.0系统后官方文档中描述说:ifconfig与route是非常陈旧的命令,取而代之的是ip命令。
 
% 那么我们先看一下老的命令使用方式:
% *************************************************************************
% ifconfig    接口  选项|地址
 
% # ifconfig  eth0  up          # 开启eth0网卡
% # ifconfig  eth0  down        # 关闭eth0网卡
% # ifconfig  eth0  -arp        # 关闭eth0网卡arp协议
% # ifconfig  eth0  promisc     # 开启eth0网卡的混合模式
% # ifconfig  eth0  mtu 1400    # 设置eth0网卡的最大传输单元为1400
% # ifconfig  eth0  192.168.0.2/24    # 设置eth0网卡IP地址
% # ifconfig  eth0  192.168.0.2  netmask 255.255.255.0    # 功能同上
 
% *************************************************************************
% 主机名:
 
% # hostname        # 查看主机名
% # hostname  butbueatiful.com    # 设置主机名为butbueatiful.com
 
% *************************************************************************
% 网关设置:
 
% route  add [-net|-host] target [netmask] gw
% route  del [-net|-host] target [netmask] gw
 
% # route add  -net 192.168.3.0/24  gw  192.168.0.254    # 设置到192.168.3.0网段的网关为192.168.0.254
% # route add  -net 192.168.3.0 netmask 255.255.255.0  gw  192.168.0.254    # 功能同上
% # route add  -host 192.168.4.4  gw  192.168.0.254    # 设置到192.168.4.4主机的网关为192.168.0.254
% #
% # route del  -net 192.168.3.0/24                        # 删除192.168.3.0网段的网关信息
% # route del  -host 192.168.4.4                        # 删除192.168.4.4主机的网关信息
% # route add default gw  192.168.0.254                # 设置默认网关为192.168.0.254
% # route del default gw  192.168.0.254                # 删除默认网关为192.168.0.254
 
% *************************************************************************
 
% 而如今官方不再推荐使用如此陈旧的命令而推荐使用 ip 这个命令,以下我们看看它的用法:
 
% ip  [选项]  操作对象{link|addr|route...}
 
% # ip link show                  # 显示网络接口信息
% # ip link set eth0 upi          # 开启网卡
% # ip link set eth0 down         # 关闭网卡
% # ip link set eth0 promisc on   # 开启网卡的混合模式
% # ip link set eth0 promisc offi # 关闭网卡的混个模式
% # ip link set eth0 txqueuelen 1200    # 设置网卡队列长度
% # ip link set eth0 mtu 1400     # 设置网卡最大传输单元
% # ip addr show                  # 显示网卡IP信息
% # ip addr add 192.168.0.1/24 dev eth0 # 设置eth0网卡IP地址192.168.0.1
% # ip addr del 192.168.0.1/24 dev eth0 # 删除eth0网卡IP地址
 
% # ip route list                 # 查看路由信息
% # ip route add 192.168.4.0/24  via  192.168.0.254 dev eth0 # 设置192.168.4.0网段的网关为192.168.0.254,数据走eth0接口
% # ip route add default via  192.168.0.254  dev eth0    # 设置默认网关为192.168.0.254
% # ip route del 192.168.4.0/24    # 删除192.168.4.0网段的网关
% # ip route del default    # 删除默认路由
 
% **************************************************************
% 接下来再看看通过文件修改网络参数:(CentOS6.2系统为例)
 
% # cat  /etc/sysconfig/network-scripts/ifcfg-eth0  
 
% DEVICE="eth0"              设备名
% NM_CONTROLLED="yes"        设备是否被NetworkManager管理
% ONBOOT="no"                开机是否启动
% HWADDR="00:0C:29:59:E2:D3" 硬件地址(MAC地址)
% TYPE=Ethernet              类型
% BOOTPROTO=none             启动协议{none|dhcp}
% IPADDR=192.168.0.1         IP地址
% PREFIX=24                  子网掩码
% GATEWAY=192.168.0.254      默认网关
% DNS1=202.106.0.20          主DNS
% DOMAIN=202.106.46.151      辅助DNS
% UUID=5fb06bd0-0bb0-7ffb-45f1-d6edd65f3e03    设备UUID编号
 
% **************************************************************
% # cat /etc/sysconfig/network
 
% HOSTNAME=butbueatiful.com    主机名
 
% **************************************************************
 
% 注意:在5.0时代DNS服务器写在 /etc/resolv.conf 文件中,但到了6.0时代DNS可以写在/etc/resolv.conf但是此时需要在 /etc/sysconfig/network-scripts/ifcfg-eth0 文件中添加 PEERDNS=no 配置,不然每次重启网卡就会重写/etc/resolv.conf文件的内容,当然了也可以直接写在 /etc/sysconfig/network-scripts/ifcfg-eth0 文件中。
 
 
% 后记:
 
% 1. 配置/etc/resolv.confg重启丢失解决方法:
 
% 一种方法是把 PEERDNS 设置为“no”。
 
% 找到网卡配置文件,位置和: /etc/sysconfig/network-scripts/ifcfg-eth 文件中加入PEERDNS 选项。可以是 0, 1, 2等等,代表不同网卡的配置文件。例如,系统上第一张网卡是eth0的话,那它的配置文件就是/etc/sysconfig/network-scripts/ifcfg-eth0 然后在文件中把 PEERDNS 改为 ‘no’.
% 例如:
 
% DEVICE=eth0
% BOOTPROTO=dhcp
% ONBOOT=yes
% TYPE=Ethernet
% PEERDNS=no
 
% 这个选项可令 /etc/resolv.conf 在系统重启后不会被重写。
 
% 另一种方法是在这个文件中增加DNS:
% 如:
% DNS1=127.0.0.1
% DNS2=8.8.8.8
 
% 2. 安全设置
 
%    我们前面说了/etc/resolv.conf和/etc/hosts被人篡改了的话, 会很危险, 那我们在设置好着2个文件后, 做一下处理, 让这2个文件默认不能直接修改, 即使root也不行, 执行如下命令:
% [root@butbueatiful ~]# chattr +i /etc/{resolv.conf,hosts}
 
% 如果我们自己想修改的时候,执行:  
% [root@butbueatiful ~]# chattr -i /etc/{resolv.conf,hosts}
 
% 然后就可以修改了,修改完了别忘记+i。
 
% 3. 网络排除思路
 
% 检查配置文件是否有错误(书写及语法错误等)
% 检查本机网络协议是否正确:# ping -c 3 127.0.0.1
% 检查本机网卡链路是否正确:# ping -c 3 192.168.0.1(本机IP地址)
% 检查网关是否正确:       # ping -c 3 192.168.0.254(网关IP地址)
% 检查外部连通性:        # ping -c 3 www.google.com.hk
% 检查硬件
\section{CentOS拨号上网}

\section{yum}


\section{安装32位的lib}

\section{NSF服务}

\section{远程安装系统}


\chapter{Ubuntu相关}









