\documentclass[10pt]{ctexrep}

\usepackage{amsmath}
\usepackage{amsthm}



\begin{document}
      


\chapter[QC]{量子化学}
\section{量化理论}

TODO.

\section{量化软件实现}
Gaussian中提取分子积分的方法如下:
\begin{verbatim}
%chk=./h2.chk
#t rhf/gen scf=conventional symm=noint noraff ExtraLinks=L316 iop(3/33=3)

H2 minimal basis set

0 1
H 1. 0. 0.
H -1. 0. 0.
H 0. -1. 0.
H 0. 1. 0.

H 0
S 1 1.00
 0.480D+00 0.100D+01
****
\end{verbatim}

\begin{verbatim}
 *** Overlap ***
                1             2             3             4
      1  0.100000D+01
      2  0.324446D-01  0.100000D+01
      3  0.180124D+00  0.180124D+00  0.100000D+01
      4  0.180124D+00  0.180124D+00  0.324446D-01  0.100000D+01
 *** Kinetic Energy ***
                1             2             3             4
      1  0.720000D+00
      2 -0.300290D-01  0.720000D+00
      3 -0.185119D-01 -0.185119D-01  0.720000D+00
      4 -0.185119D-01 -0.185119D-01 -0.300290D-01  0.720000D+00

...

 ***** Potential Energy *****
                1             2             3             4
      1  0.211838D+01
      2  0.680691D-01  0.211838D+01
      3  0.372883D+00  0.372883D+00  0.211838D+01
      4  0.372883D+00  0.372883D+00  0.680691D-01  0.211838D+01
 ****** Core Hamiltonian ******
                1             2             3             4
      1 -0.139838D+01
      2 -0.980981D-01 -0.139838D+01
      3 -0.391395D+00 -0.391395D+00 -0.139838D+01
      4 -0.391395D+00 -0.391395D+00 -0.980981D-01 -0.139838D+01


...







 *** Dumping Two-Electron integrals ***



 ISMode= 0 Mode= 1 IBase=         1 IBasD=         1    262145
 DBase=         0 DBasD=         0         0 IReset=         2    262139
 IntCnt=         0 ITotal=        55 NWIIB=    262144 ISym2E=0
 I=  4 J=  3 K=  2 L=  1 Int=  0.822923860669D-03
 I=  4 J=  1 K=  3 L=  2 Int=  0.120330383547D-01
 I=  4 J=  2 K=  3 L=  1 Int=  0.120330383547D-01
 I=  4 J=  4 K=  4 L=  4 Int=  0.781764019045D+00
 I=  4 J=  4 K=  4 L=  3 Int=  0.160685510571D-01
 I=  4 J=  4 K=  4 L=  2 Int=  0.109126261413D+00
 I=  4 J=  4 K=  4 L=  1 Int=  0.109126261413D+00
 I=  4 J=  3 K=  4 L=  3 Int=  0.822923860669D-03
 I=  4 J=  4 K=  3 L=  3 Int=  0.264532250132D+00
 I=  4 J=  3 K=  4 L=  2 Int=  0.354055434816D-02
 I=  4 J=  4 K=  3 L=  2 Int=  0.600780522778D-01
 I=  4 J=  3 K=  4 L=  1 Int=  0.354055434816D-02
 I=  4 J=  4 K=  3 L=  1 Int=  0.600780522778D-01
 I=  4 J=  2 K=  4 L=  2 Int=  0.253639954401D-01
 I=  4 J=  4 K=  2 L=  2 Int=  0.370879913132D+00
 I=  4 J=  2 K=  4 L=  1 Int=  0.160685510571D-01
 I=  4 J=  4 K=  2 L=  1 Int=  0.160685510571D-01
 I=  4 J=  1 K=  4 L=  1 Int=  0.253639954401D-01
 I=  4 J=  4 K=  1 L=  1 Int=  0.370879913132D+00
 I=  4 J=  3 K=  3 L=  3 Int=  0.160685510571D-01
 I=  4 J=  2 K=  3 L=  3 Int=  0.600780522778D-01
 I=  4 J=  3 K=  3 L=  2 Int=  0.354055434816D-02
 I=  4 J=  1 K=  3 L=  3 Int=  0.600780522778D-01
 I=  4 J=  3 K=  3 L=  1 Int=  0.354055434816D-02
 I=  4 J=  2 K=  3 L=  2 Int=  0.160685510571D-01
 I=  4 J=  3 K=  2 L=  2 Int=  0.160685510571D-01
 I=  4 J=  1 K=  3 L=  1 Int=  0.160685510571D-01
 I=  4 J=  3 K=  1 L=  1 Int=  0.160685510571D-01
 I=  4 J=  2 K=  2 L=  2 Int=  0.109126261413D+00
 I=  4 J=  1 K=  2 L=  2 Int=  0.600780522778D-01
 I=  4 J=  2 K=  2 L=  1 Int=  0.354055434816D-02
 I=  4 J=  1 K=  2 L=  1 Int=  0.354055434816D-02
 I=  4 J=  2 K=  1 L=  1 Int=  0.600780522778D-01
 I=  4 J=  1 K=  1 L=  1 Int=  0.109126261413D+00
 I=  3 J=  3 K=  3 L=  3 Int=  0.781764019045D+00
 I=  3 J=  3 K=  3 L=  2 Int=  0.109126261413D+00
 I=  3 J=  3 K=  3 L=  1 Int=  0.109126261413D+00
 I=  3 J=  2 K=  3 L=  2 Int=  0.253639954401D-01
 I=  3 J=  3 K=  2 L=  2 Int=  0.370879913132D+00
 I=  3 J=  2 K=  3 L=  1 Int=  0.160685510571D-01
 I=  3 J=  3 K=  2 L=  1 Int=  0.160685510571D-01
 I=  3 J=  1 K=  3 L=  1 Int=  0.253639954401D-01
 I=  3 J=  3 K=  1 L=  1 Int=  0.370879913132D+00
 I=  3 J=  2 K=  2 L=  2 Int=  0.109126261413D+00
 I=  3 J=  1 K=  2 L=  2 Int=  0.600780522778D-01
 I=  3 J=  2 K=  2 L=  1 Int=  0.354055434816D-02
 I=  3 J=  1 K=  2 L=  1 Int=  0.354055434816D-02
 I=  3 J=  2 K=  1 L=  1 Int=  0.600780522778D-01
 I=  3 J=  1 K=  1 L=  1 Int=  0.109126261413D+00
 I=  2 J=  2 K=  2 L=  2 Int=  0.781764019045D+00
 I=  2 J=  2 K=  2 L=  1 Int=  0.160685510571D-01
 I=  2 J=  1 K=  2 L=  1 Int=  0.822923860669D-03
 I=  2 J=  2 K=  1 L=  1 Int=  0.264532250132D+00
 I=  2 J=  1 K=  1 L=  1 Int=  0.160685510571D-01
 I=  1 J=  1 K=  1 L=  1 Int=  0.781764019045D+00
\end{verbatim}


% 下面时对pyQuante的解读,有乱码改日处理
% \section{¶ÔPyQuanteÖйØÓÚ»ù×é»ý·Ö´úÂëµÄ×¢ÊÍ}
% \subsection{¸ÅÊö}
% ÒÔÏÂ×¢½â»ùÓÚPyQuante-1.6.5°æ±¾£¬Ô´ÎļþλÓÚĿ¼£º\verb"~\PyQuante\"¡£Ò»Ï¶Ôÿ¸öÎļþÒÔÒ»¸ösubsection½øÐÐÌÖÂÛ¡£







% \subsection{PGBF.py}
% °üº¬¹ØÓÚԭʼ¸ß˹»ùº¯ÊýµÄ»ù±¾²Ù×÷\\
% ²Î¿¼ÎÄÏ×£º\\
% 'Gaussian Expansion Methods for Molecular Orbitals.' H. Taketa, S. Huzinaga, and K. O-ohata. H. Phys. Soc. Japan, 21, 2313, 1966.[THO paper]


% ¸ß˹º¯ÊýµÄ¶¨ÒåΪ£º
% \begin{equation}
% g(x,y,z)=Ax^{i}y^{j}z^{k}\exp{[-a(r-r_0)^2]}
% \end{equation}
% ±¾Ä£¿é°üº¬Ò»Ï·½·¨£º\\
% overlap(g'): ¼ÆËãgÓëg'µÄÖصþ¾ØÕó: Int(g*g')\\
% kinetic(g'): ¼ÆËãgÓëg'µÄ¶¯ÄÜ»ý·Ö£ºInt(G*lapl(G')), ÆäÖÐlapl±íʾÀ­ÆÕÀ­Ë¹Ëã·û.\\
% nuclear(g',r): ¼ÆËãºËÎüÒýÄÜ»ý·Ö\\
% Int(g*(1/r)*g'). Only programmed for 1s gaussians.\\
% coulomb(g,g',g'',g'''): Compute the two-electron colombic repulsion\\
% integral Int(g(1)g'(1)(1/r12)g''(2)g'''(2)). \\


% \section{量化软件实现}

% \subsection{CGBF.py}
% °üº¬¹ØÓÚÊÕËõ¸ß˹»ùº¯ÊýµÄ»ù±¾²Ù×÷


% \subsection{Ints.py}
% »ý·ÖÎļþ


% \subsection{pyints.py}
% Python implementations of work functions for Gaussian integrals in the PyQuante package.\\
% The equations herein are based upon\\
% 'Gaussian Expansion Methods for Molecular Orbitals.' H. Taketa, S. Huzinaga, and K. O-ohata. H. Phys. Soc. Japan, 21, 2313, 1966.[THO paper]

\section{量化软件使用心得:Gaussian}
\subsection{Gaussian中优化不收敛的解决办法}

\section{量化软件使用心得:Material Studio}
\subsection{MS xxx}

\end{document}
